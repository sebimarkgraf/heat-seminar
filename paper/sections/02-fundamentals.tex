\section{Fundamentals}
\label{sec:fundamentals}

\subsection{Distributed Array Computation}
\label{ssec:distributed_array_computation}

\subsection{HEAT}
\label{ssec:heat}
\gls{HeAT} is introduced in \cite{krajsek_helmholtz_nodate}

\subsection{Remote Sensing and Local Climate Zone Classification}
\label{ssec:remote_sensing_and_local_climate_zone_classification}

\subsection{SO2Sat}
\citeauthor{zhu_so2sat_2019} presented the SO2Sat dataset in \cite{zhu_so2sat_2019}. They describe SO2Sat as a \enquote{valuable benchmark dataset [...], which consists of local climate zone (LCZ) labels of about half a million [...] image patches}.
These images are taken from Sentinel-1 and Sentinel-2. Each image is 32 by 32 pixels and contains 8 channels for Sentinel-1 and 10 channels for Sentinel-2.
After preprocessing these images they were given to domain experts for labelling which follwed a \enquote{carefully designed labelling work flow}.
Through the careful work the dataset achieved a \enquote{overall confidence of 85\%}.
The annotations contains the 17 LCZ classes.
The regions of the dataset are 52 cities.

\subsection{Clustering Algorithms}
\label{ssec:clustering_algorithms}


\subsection{Clustering Evaluation Metrics}
\label{ssec:clustering_evaluation_metrics}

Evaluating clusters is more complicated than evaluating classification or other supervised machine learning tasks.
Unsupervised tasks can be evaluating using interal or external metrics. While, internal metrics are derived from the algorithm itself e.g. Cluster purity, external metrics need labels for the dataset.

Due to in this work used dataset having labels, this section is going to focus on external metrics.

Adjusted Rand Index

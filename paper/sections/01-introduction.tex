
\section{Introduction}
\label{sec:intro}
Over the last decade not only the computing power is growing exponentially, but the data sets are growing extremely fast as well.
Collecting data is becoming easier, especially through the vast amount of online services.
Although, this allows machine learning algorithms to become more and more precise, there are downsides arising as well with
the increased size.

To store the datasets the storage capacity needs to increase in the same amount. More importantly, the computer power needs to increase
as well. But just using raw power does not help which leaves us with using efficient algorithms and leveraging the underlying comput power optimally.
Especially, when using \gls{GPU} this is not an easy task. To help with the task many libraries were born e.g. PyTorch.
They provide mathematical computations efficiently implemented and leverage an \gls{GPU} if available.

Still, single machines are reaching their limit when working with huge data sets. The solution lies in using multiple computing nodes.
But, while the libraries already provide excellent single node support, they do not provide an easy interface to work on a distributed system.
This leaves the developer with the task of managing different nodes and distribute the data correctly on the compute node.

The library developed by the Helmholtz Community \gls{HeAT} aims to solve this problem.
Providing an \gls{numpy} like interfface the library wraps \gls{PyTorch} and \gls{MPI} to allow the developer to write his script on a single computer
and perform it on a distributed compute cluster without worrying about the correct synchronization.

Another rising problems lies in the tedious work of labelling the data set.
To use these data sets for classification or other machine learning purposes instance labels are needed.
These labels needs to be precise and often need to be created manually.
Due to the sheer amount of data that needs to be labelled, this process is expensive and takes a lot of time.

Unsupervised learning is the category of machine learning algorithms that do not use labels for their purpose. They try to
use the inherent structure of the dataset itself to find useful information and insights.
A well known family of algorithms are the clustering algorithms. They are an unsupervised pendant to classification algorithms
and split the dataset in different clusters according to the features.
When combined with expert knowledge this could allow to classify datasets without labelling them first.

When combining these two approaches, it allows to use huge data sets without the tedious work of labelling and with minimal development effort.
The goal of this paper is to evaluate the approach on remote sensing data which can only be labelled with a lot of manual effort.

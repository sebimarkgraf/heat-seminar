\begin{abstract}
  Using unsupervised learning for datasets allows to extract information
  without the tedious task of labeling. Especially, on high-volume datasets
  this provides a potential time and cost saving benefit.

  For processing high-volume datasets the power of a single machine is
  not sufficient. By distributing the work across the memory and compute power of multiple machines,
  it is possible to load bigger datasets and speedup the computation.
  But due to the difficulty of writing distributed tasks this should be encapsulated into libraries that transparently split the
  data and work.

  In this work we try to use the HPC library HeAT on remote sensing data to find a clustering
  without using the labels. The result of the clustering is then compared with the labels.
  Additionally, the scaling behavior of HeAT depending on the number of computing nodes is profiled.
\end{abstract}

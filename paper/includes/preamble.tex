
% use nicer font for code
\usepackage[zerostyle=b,scaled=.75]{newtxtt}

% for demonstration purposes
\usepackage{mwe}

\usepackage[T1]{fontenc}
\usepackage[utf8]{inputenc} %support umlauts in the input

\usepackage{amsmath}

\usepackage{graphicx}

%Set English as language and allow to write hyphenated"=words
\usepackage[ngerman,main=english]{babel}
%Hint by http://tex.stackexchange.com/a/321066/9075 -> enable "= as dashes
\addto\extrasenglish{\languageshorthands{ngerman}\useshorthands{"}}

% backticks (`) are rendered as such in verbatim environment. See https://tex.stackexchange.com/a/341057/9075 for details.
\usepackage{upquote}

%extended enumerate, such as \begin{compactenum}
\usepackage{paralist}

%for easy quotations: \enquote{text}
\usepackage{csquotes}

%enable margin kerning
\RequirePackage{iftex}
\ifPDFTeX
  \RequirePackage[%
    final,%
    expansion=alltext,%
    protrusion=alltext-nott]{microtype}%
\else
  \RequirePackage[%
    final,%
    protrusion=alltext-nott]{microtype}%
\fi%
% \texttt{test -- test} keeps the "--" as "--" (and does not convert it to an en dash)
\DisableLigatures{encoding = T1, family = tt* }

%tweak \url{...}
\usepackage{url}
%\urlstyle{same}
%improve wrapping of URLs - hint by http://tex.stackexchange.com/a/10419/9075
\makeatletter
\g@addto@macro{\UrlBreaks}{\UrlOrds}
\makeatother
%nicer // - solution by http://tex.stackexchange.com/a/98470/9075
%DO NOT ACTIVATE -> prevents line breaks
%\makeatletter
%\def\Url@twoslashes{\mathchar`\/\@ifnextchar/{\kern-.2em}{}}
%\g@addto@macro\UrlSpecials{\do\/{\Url@twoslashes}}
%\makeatother

% Diagonal lines in a table - http://tex.stackexchange.com/questions/17745/diagonal-lines-in-table-cell
% Slashbox is not available in texlive (due to licensing) and also gives bad results. This, we use diagbox
%\usepackage{diagbox}

\usepackage{booktabs}

% Required for package pdfcomment later
\usepackage{xcolor}

% For listings
\usepackage{listings}
\lstset{%
  basicstyle=\ttfamily,%
  columns=fixed,%
  basewidth=.5em,%
  xleftmargin=0.5cm,%
  captionpos=b}%
% Fix counter as described at https://tex.stackexchange.com/a/28334/9075
\usepackage{chngcntr}
\AtBeginDocument{\counterwithout{lstlisting}{section}}

% Compatibility of packages minted and listings with respect to the numbering of "Listing" caption
% Source: https://tex.stackexchange.com/a/269510/9075
\AtBeginEnvironment{listing}{\setcounter{listing}{\value{lstlisting}}}
\AtEndEnvironment{listing}{\stepcounter{lstlisting}}

% Enable nice comments
\usepackage{pdfcomment}
%
\newcommand{\commentontext}[2]{\colorbox{yellow!60}{#1}\pdfcomment[color={0.234 0.867 0.211},hoffset=-6pt,voffset=10pt,opacity=0.5]{#2}}
\newcommand{\commentatside}[1]{\pdfcomment[color={0.045 0.278 0.643},icon=Note]{#1}}
%
% Compatibility with packages todo, easy-todo, todonotes
\newcommand{\todo}[1]{\commentatside{#1}}
% Compatiblity with package fixmetodonotes
\newcommand{\TODO}[1]{\commentatside{#1}}

% Bibliopgraphy enhancements
%  - enable \cite[prenote][]{ref}
%  - enable \cite{ref1,ref2}
% Alternative: \usepackage{cite}, which enables \cite{ref1, ref2} only (otherwise: Error message: "White space in argument")
%
% Doc: http://texdoc.net/natbib
\ifCLASSOPTIONcompsoc
  % IEEE Computer Society needs nocompress option at cite.sty
  % natbib includes the same functionality
  \usepackage[%
    square,        % for square brackets
    comma,         % use commas as separators
    numbers,       % for numerical citations;
    sort           % orders multiple citations into the sequence in which they appear in the list of references;
    %sort&compress % as sort but in addition multiple numerical citations
                   % are compressed if possible (as 3-6, 15);
  ]{natbib}
\else
  % normal IEEE
  \usepackage[%
    square,        % for square brackets
    comma,         % use commas as separators
    numbers,       % for numerical citations;
    %sort           % orders multiple citations into the sequence in which they appear in the list of references;
    sort&compress % as sort but in addition multiple numerical citations
                   % are compressed if possible (as 3-6, 15);
  ]{natbib}
\fi
% Same fontsize as without natbib
\renewcommand{\bibfont}{\normalfont\footnotesize}
% Enable hyperlinked author names in the case of \citet
% Source: https://tex.stackexchange.com/a/76075/9075
\usepackage{etoolbox}
\makeatletter
\patchcmd{\NAT@test}{\else \NAT@nm}{\else \NAT@hyper@{\NAT@nm}}{}{}
\makeatother

% Enable that parameters of \cref{}, \ref{}, \cite{}, ... are linked so that a reader can click on the number an jump to the target in the document
\usepackage{hyperref}
% Enable hyperref without colors and without bookmarks
\hypersetup{hidelinks,
  colorlinks=true,
  allcolors=black,
  pdfstartview=Fit,
  breaklinks=true}
%
% Enable correct jumping to figures when referencing
\usepackage[all]{hypcap}

%enable \cref{...} and \Cref{...} instead of \ref: Type of reference included in the link
\usepackage[capitalise,nameinlink]{cleveref}
\crefname{lstlisting}{\lstlistingname}{\lstlistingname}
\Crefname{lstlisting}{Listing}{Listings}

\usepackage[newfloat]{minted}

% Line numbers not flowing out of the margin
\setminted{numbersep=5pt, xleftmargin=12pt}

\usemintedstyle{bw} %black and white style
%\usemintedstyle{vs} %visual studio
%\usemintedstyle{friendlygrayscale} % custom style - submitted as pull request https://bitbucket.org/birkenfeld/pygments-main/pull-requests/748/add-style-friendly-grayscale/diff
%\usemintedstyle{friendly}
%\usemintedstyle{eclipse} %http://www.jevon.org/wiki/Eclipse_Pygments_Style
%\usemintedstyle{autumn}
%\usemintedstyle{rrt}
%\usemintedstyle{borland}

% We need to load caption to have a bold font on the label
% The other parameters mimic the layout of the LNCS class
\usepackage[labelfont=bf,font=small,skip=4pt]{caption}
\SetupFloatingEnvironment{listing}{name=Listing,within=none}
%
%Intermediate solution for hyperlinked refs. See https://tex.stackexchange.com/q/132420/9075 for more information.
\newcommand{\Vlabel}[1]{\label[line]{#1}\hypertarget{#1}{}}
\newcommand{\lref}[1]{\hyperlink{#1}{\FancyVerbLineautorefname~\ref*{#1}}}

\usepackage{xspace}
%\newcommand{\eg}{e.\,g.\xspace}
%\newcommand{\ie}{i.\,e.\xspace}
\newcommand{\eg}{e.\,g.,\ }
\newcommand{\ie}{i.\,e.,\ }

%introduce \powerset - hint by http://matheplanet.com/matheplanet/nuke/html/viewtopic.php?topic=136492&post_id=997377
\DeclareFontFamily{U}{MnSymbolC}{}
\DeclareSymbolFont{MnSyC}{U}{MnSymbolC}{m}{n}
\DeclareFontShape{U}{MnSymbolC}{m}{n}{
  <-6>    MnSymbolC5
  <6-7>   MnSymbolC6
  <7-8>   MnSymbolC7
  <8-9>   MnSymbolC8
  <9-10>  MnSymbolC9
  <10-12> MnSymbolC10
  <12->   MnSymbolC12%
}{}
\DeclareMathSymbol{\powerset}{\mathord}{MnSyC}{180}

% *** SUBFIGURE PACKAGES ***
\ifCLASSOPTIONcompsoc
  \usepackage[caption=false,font=footnotesize,labelfont=sf,textfont=sf]{subfig}
\else
  \usepackage[caption=false,font=footnotesize]{subfig}
\fi

% correct bad hyphenation here
\hyphenation{op-tical net-works semi-conduc-tor}

\usepackage{glossaries}


\usepackage[linesnumbered,lined,boxed,commentsnumbered]{algorithm2e}
